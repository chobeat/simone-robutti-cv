\documentclass{tccv}
\usepackage[english]{babel}
\usepackage[utf8]{inputenc}
\usepackage[T1]{fontenc}
\begin{document}

\part{Simone Robutti}

\begin{eventlist}

\item{Nov 2016- Present}
     {MotionLogic, Berlin}
     {Data Engineer}

Development and testing of the company's core product: a system to ingest and process cellular network signals to extract complex informations through ad-hoc models and traditional aggregations. My role as a Data Engineer covered many different tasks over time: productization of existing algorithms developed by the Data Science Team, development of new models and heuristics on my own, development and maintenance of the pipeline, development of a data quality monitoring system, production and improvement of geospatial support data, performance optimization at scale.


\noindent\hfil\rule{0.3\textwidth}{.4pt}

\item{Feb 2016 - Nov 2016}
{Databiz - Radicalbit, Milan}
{Machine Learning Engineer}

As the only Machine Learning Engineer, my work involved designing new solutions for our customers to make use of Flink, core component of our distribution. Also I've been involved in the open-source development of FlinkML and FlinkJPMML. I had the possibility to implement many different algorithms from scratch and experiment with the many possibilities offered by the framework to carry out ML tasks at scale. Other relevant components of the distribution were: Cassandra, Kafka, Zeppelin, Alluxio.

\noindent\hfil\rule{0.3\textwidth}{.4pt}

\item{Apr 2015 - Feb 2016}
{Databiz, Milan}
{Machine Learning Engineer}

Design and implementation of a near-real-time scoring streaming system. The end-goal was to provide the user with a flexible system based on Apache Spark to design ML pipelines for data preparation and batch learning on Knime and deploy the models created for scoring in streaming. The project involved a model repository with a front-end component, a data validation and reconciliation process. Other tools employed in the project were: HBase, JPMML, Hive, Impala.

\end{eventlist}

\personal
    [https://github.com/chobeat/]
    {Keibelstr. 37, 10178, Berlin}
    {simone.robutti@gmail.com}

\section{Education}

\begin{yearlist}

\item[Master Degree]{2012-2015}
     {Computer Science}
     {Università degli Studi di Milano}

\item[Bachelor Degree]{2007-2012}
{Computer Science}
{Università degli Studi di Milano}

\item[High School Diploma]{2002-2007}
{Scientific specialization}
{Liceo Galileo Galilei, Alessandria}



\end{yearlist}

\section{Public projects}

\begin{yearlist}

\item{2017}
     {scala-json-feed (\href{https://github.com/chobeat/scala-json-feed}{github.com/chobeat/scala-json-feed})}
     {Scala library for JSON Feed}

\item{2016}
     {MIDFMBM (\href{https://github.com/chobeat/MIDFMBM}{github.com/chobeat/MIDFMBM})}
     {Experimental videogame in Scala}

\end{yearlist}

\section{Technical skills}

\begin{factlist}

\item{Work}
     {Scala, Python, Spark, Flink, SQL, MongoDB, Cassandra, InfluxDB, Grafana, pandas, matplotlib, ALS, LSH, git, yarn, PMML, JSON, XML, maven, sbt, setuptools, tox}

\item{University}
{SVM, Java, Genetic Algorithms, Data Mining, Sentiment Analysis, \LaTeX}

\item{Studying}
     {Deep Learning, Reinforcement Learning, Keras, CNN, RNN, dl4j/rl4j, Clojure, Docker}


\end{factlist}

\end{document}
